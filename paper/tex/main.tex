\documentclass[12pt, letterpaper, preprint, comicneue]{aastex63}
\usepackage[T1]{fontenc}
\usepackage{fontawesome}
\usepackage{color}
\usepackage{amsmath}
\usepackage{natbib}
\usepackage{ctable}
\usepackage{bm}
\usepackage[normalem]{ulem} 
\usepackage{xspace}
\usepackage{paralist}


% typesetting shih
\linespread{1.08} % close to 10/13 spacing
\setlength{\parindent}{1.08\baselineskip} % Bringhurst
\setlength{\parskip}{0ex}
\let\oldbibliography\thebibliography % killin' me.
\renewcommand{\thebibliography}[1]{%
  \oldbibliography{#1}%
  \setlength{\itemsep}{0pt}%
  \setlength{\parsep}{0pt}%
  \setlength{\parskip}{0pt}%
  \setlength{\bibsep}{0ex}
  \raggedright
}
\setlength{\footnotesep}{0ex} % seriously?

% citation alias

% math shih
\newcommand{\setof}[1]{\left\{{#1}\right\}}
\newcommand{\given}{\,|\,}
\newcommand{\lss}{{\small{LSS}}\xspace}

\newcommand{\Om}{\Omega_{\rm m}} 
\newcommand{\Ob}{\Omega_{\rm b}} 
\newcommand{\OL}{\Omega_\Lambda}
\newcommand{\smnu}{M_\nu}
\newcommand{\sig}{\sigma_8} 
\newcommand{\mmin}{M_{\rm min}}
\newcommand{\BOk}{\widehat{B}_0} 
\newcommand{\hmpc}{\,h/\mathrm{Mpc}}
\newcommand{\bfi}[1]{\textbf{\textit{#1}}}
\newcommand{\parti}[1]{\frac{\partial #1}{\partial \theta_i}}
\newcommand{\partj}[1]{\frac{\partial #1}{\partial \theta_j}}
\newcommand{\mpc}{{\rm Mpc}}
\newcommand{\eg}{\emph{e.g.}}
\newcommand{\ie}{\emph{i.e.}}

\let\oldAA\AA
\renewcommand{\AA}{\text{\normalfont\oldAA}}
% cmds for this paper 
\newcommand{\gr}{g{-}r}
\newcommand{\fnuv}{FUV{-}NUV}
\newcommand{\sfr}{{\rm SFR}}
\newcommand{\ssfr}{{\rm SSFR}}
\newcommand{\xobs}{\bfi{x}_{\rm obs}}
\newcommand{\btheta}{\boldsymbol{\theta}}
\newcommand{\bphi}{\boldsymbol{\phi}}
\newcommand{\specialcell}[2][c]{%
  \begin{tabular}[#1]{@{}c@{}}#2\end{tabular}}
% text shih
\newcommand{\foreign}[1]{\textsl{#1}}
\newcommand{\etal}{\foreign{et~al.}}
\newcommand{\opcit}{\foreign{Op.~cit.}}
\newcommand{\documentname}{\textsl{Article}}
\newcommand{\equationname}{equation}
\newcommand{\bitem}{\begin{itemize}}
\newcommand{\eitem}{\end{itemize}}
\newcommand{\beq}{\begin{equation}}
\newcommand{\eeq}{\end{equation}}

\newcommand{\github}{\href{https://github.com/changhoonhahn/DELi/}{\faGithub}}


\newcommand{\sedflow}{{\sc SEDflow}}
%% collaborating
\newcommand{\todo}[1]{\marginpar{\color{red}TODO}{\color{red}#1}}
\definecolor{orange}{rgb}{1,0.5,0}
\newcommand{\chedit}[1]{{\color{orange}#1}}
\newcommand{\peter}[1]{{\color{red}#1}}

\begin{document} \sloppy\sloppypar\frenchspacing 

\title{}

\newcounter{affilcounter}
\author[0000-0003-1197-0902]{ChangHoon Hahn}
\altaffiliation{changhoon.hahn@princeton.edu.com}
\affil{Department of Astrophysical Sciences, Princeton University, Princeton NJ 08544, USA} 


\begin{abstract}
    It's standard practice to model the photometry and spectra of galaxies
    using stellar population synthesis (SPS) models that include contributions from
    nebular continuum and emission. 
    In this work, we use spectra from the DESI Bright Galaxy Survey EDR to
    assess whether the predicted nebular continuum and emission sufficiently
    describe the observed emission line. 
    We first estimate the distribution of emission lines given continuum using
    a spectral encorder and normalziing flow. 
    We then compare this distribution to the emission line distribution
    predicted by the prospector SPS model for a given continuum. 
    Our comparison reveals that for low-redshift galaxies, the SPS model predict
    overly conservative(?) distributions of emission lines. 
    In other words, for a given set of parameters, the SPS model does not
    predict the full range of emission lines. 
    For SED modeling, this limitation can produce overconfident posteriors of
    galaxy properties. 
    We also demonstrate that it can bias the posteriors of e.g. stellar mass or
    sfr by some dex. 
    \github
\end{abstract}
\keywords{galaxies: evolution -- galaxies: statistics}

\section{Methods}\label{sec:methods}
Our goal is to estimate the likelihood distribution of emission line fluxes given
parameters of our SPS model: $p(\hat{f}_{\rm e} \given \theta)$. 
We estimate this likelihood in two complementary ways using a model-driven and
a data-driven approach.


\subsection{SPS Model Driven Estimate}\label{sec:model}
For the model driven, 
\begin{equation}\label{eq:model_driven}
    p(\hat{f}_{\rm e} \given \theta) = \int p(\hat{f}_{\rm e}\given f_{\rm
    e})\,p(f_{\rm e}\given \theta)\,{\rm d}f_{\rm e}.
\end{equation} 
$p(\hat{f}_{\rm e}\given f_{\rm e})$ corresponds to the noise model on the
emission line fluxes and $p(f_{\rm e}\given \theta)$ corresponds to the
emission line flux prediction of the SPS model. 


\subsection{Data-Driven Estimate}\label{sec:model}
For the data-driven 
\begin{equation}\label{eq:data_driven}
    p(\hat{f}_{\rm e} \given \theta) = \int p(\hat{f}_{\rm e}\given \hat{f}_{\rm
    c})\,p(\hat{f}_{\rm c}\given \theta)\,{\rm d}f_{\rm c} 
\end{equation} 
$p(\hat{f}_{\rm e}\given \hat{f}_{\rm c})$ corresponds to the the distribution
of emission line fluxes given a galaxy's continuum spectrum. 
Meanwhile, $p(\hat{f}_{\rm c}\given \theta)$ represents the likelihood of the
continuum spectrum for a given set of SPS parameters. 
In practice, we estimate $p(\hat{f}_{\rm e}\given \hat{f}_{\rm c}) \approx
p(\hat{f}_{\rm e}\given s_{\rm c})$, where $s_{\rm c}$ corresponds to the
latent representation (compression) of $\hat{f}_{\rm c}$.  



\section*{Acknowledgements}
It's a pleasure to thank 

for valuable discussions and comments.
This work was supported by the AI Accelerator program of the Schmidt Futures Foundation.

\appendix
%\bibliographystyle{mnras}
%\bibliography{deli} 
\end{document}
