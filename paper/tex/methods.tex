\section{Methods}\label{sec:methods}
Our goal is to estimate the likelihood distribution of emission line fluxes given
parameters of our SPS model: $p(\hat{f}_{\rm e} \given \theta)$. 
We estimate this likelihood in two complementary ways using a model-driven and
a data-driven approach.


\subsection{SPS Model Driven Estimate}\label{sec:model}
For the model driven, 
\begin{equation}\label{eq:model_driven}
    p(\hat{f}_{\rm e} \given \theta) = \int p(\hat{f}_{\rm e}\given f_{\rm
    e})\,p(f_{\rm e}\given \theta)\,{\rm d}f_{\rm e}.
\end{equation} 
$p(\hat{f}_{\rm e}\given f_{\rm e})$ corresponds to the noise model on the
emission line fluxes and $p(f_{\rm e}\given \theta)$ corresponds to the
emission line flux prediction of the SPS model. 


\subsection{Data-Driven Estimate}\label{sec:model}
For the data-driven 
\begin{equation}\label{eq:data_driven}
    p(\hat{f}_{\rm e} \given \theta) = \int p(\hat{f}_{\rm e}\given \hat{f}_{\rm
    c})\,p(\hat{f}_{\rm c}\given \theta)\,{\rm d}f_{\rm c} 
\end{equation} 
$p(\hat{f}_{\rm e}\given \hat{f}_{\rm c})$ corresponds to the the distribution
of emission line fluxes given a galaxy's continuum spectrum. 
Meanwhile, $p(\hat{f}_{\rm c}\given \theta)$ represents the likelihood of the
continuum spectrum for a given set of SPS parameters. 
In practice, we estimate $p(\hat{f}_{\rm e}\given \hat{f}_{\rm c}) \approx
p(\hat{f}_{\rm e}\given s_{\rm c})$, where $s_{\rm c}$ corresponds to the
latent representation (compression) of $\hat{f}_{\rm c}$.  

